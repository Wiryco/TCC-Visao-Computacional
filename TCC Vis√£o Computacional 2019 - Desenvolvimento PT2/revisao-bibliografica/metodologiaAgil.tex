\section{{DESENVOLVIMENTO ÁGIL DE \textit{SOFTWARE}}}

Quando se pensa em desenvolvimento de \textit{software}, deve-se reconhecer que o processo é bem instável e com baixa previsibilidade, se tornando algo complicado de estabelecer métricas a serem seguidas. Vários pontos de interferência pode intervir para que um \textit{software} não possa ser desenvolvido da maneira correta, como por exemplo uma má equipe de trabalho, nos quais os integrantes não possuem uma boa convivência entre si, ou uma má metodologia de desenvolvimento sem uma linha cronológica a ser seguida. Reconhecer que este é um grande desafio é algo sensato de se fazer. No entanto existem mecanismos de correção para melhorar o processo de desenvolvimento.

A metodologia ágil surgiu para organizar esses processos de desenvolvimento de \textit{software}, elaborando uma padronização nos projetos para que seja possível otimizar os fluxos de trabalho e melhoras a produtividade do projeto. Segundo \citeonline{SOARES2004}, a principal característica da metodologia ágil e que ela pode ser adaptada, ao invés de ser preditiva. Ou seja, se ocorrer algum problema no decorrer do desenvolvimento, a própria metodologia é flexível o bastante para contornar a situação e prosseguir com o desenvolvimento do projeto. Portanto, uma empresa pode facilmente criar a sua própria metodologia de trabalho, seguindo a sua experiência empresarial, analisando os seus acertos e erros para elaborar um procedimento que se adéqua as suas necessidades. 

\citeonline{SOARES2004} continua exemplificando que a metodologia ágil trabalha com constantes \textit{feedbacks} e reuniões, nas quais permitem aos membros da equipe expor as facilidades e dificuldades de suas tarefas, bem como o seu status de desenvolvimento. A partir destes \textit{feedbacks}, o gestor pode traçar a melhor maneira de organizar a equipe para que os membros com maior experiência deem apoio nas dificuldades apresentadas pelos outros integrantes da equipe. Outro grande motivo expressado no contexto é o fato ocorrer entregas constantes de partes operacionais do software. Desta maneira, o usuário final do software pode ter uma prévia de como o sistema está sendo desenvolvido, bem como suas funcionalidades e \textit{design}, podendo solicitar alguma possível alteração ou identificar algum problema antes da implantação oficial dos modulos.


\subsection{\textbf{\textit{Kanban}}}

De forma a complementar o assunto, o \textit{Kanban} é um método ágil de desenvolvimento de software que permite a interação de várias áreas e membros do projeto por meio de cartões que contem o progresso de cada atividade. Cada cartão contem uma instrução a ser seguida pela área ou pelo integrante no qual foi designado para a atividade. Sendo assim, seu principal foco e fornecer um trabalho progressivo, apresentando as evoluções e dificuldades de forma clara e transparente, favorecendo uma cultura de melhoria contínua \cite{KANBAN2014}.

Sendo assim, o \textit{Kanban} tem um grande potencial em trabalho conjunto para a finalização de um item em especifico, justamente para que não ocorra nenhum gargalo na entrega de um item essencial para o trabalho do integrante ou da área seguinte. Outra grande característica é evitar ou diminuir o índice de trabalhos repetitivos que, por um eventual descuido, possa acontecer de desenvolvedores realizarem a mesma codificação de uma mesma função ou \textit{API - Application Programming Interface} (Interface de Programação de Aplicações), por exemplo.

O \textit{Kanban} foi criado pelo vice presidente da \textit{Toyota Motor Company}, o sr. Taiichi Ohno, no qual teve como principal objetivo o aumento do valor agregado entregue nas atividades de cada colaborador de sua equipe. Com este pensamento,  Ohno concluiu que as pilhas de materiais estocados e as filas de de espera era um “dinheiro parado” que a empresa \textit{Toyota} estava desperdiçando. Portando,   Ohno uniu os princípios do método \textit{just in time} (Determina que tudo deve ser feito na hora exata)  juntamente com o \textit{Jidoka} (Determina que corrigir o problema em si não é o bastante, e sim corrigir a origem do problema) para elaborar um metodo mais aprimorado de organização dentro da empresa \textit{Toyota}, denominado \textit{Kanban} \cite{TOYOTA1977}.

Atualmente, a metodologia ágil \textit{Kanban} é utilizada em diversas empresas para realizar o controle de desempenho de diversas área de atuação.
\subsection{\textbf{\textit{Trello}}}