\subsubsection{\textit{Framework}}

Atualmente, os projetos de desenvolvimento de \textit{software} aumentou consideravelmente. Devido a isso, programadores se deparam com um excesso de funções similares dentro dos vários projetos desenvolvidos no decorrer do tempo. Sendo assim, surgiu a necessidade de reutilizar códigos para poupar tempo. 

De acordo com \citeonline{JOHNSON97}, \textit{framework} são estruturas desenvolvidas com o objetivo de reutilizar tudo ou parte de um sistema representado por um conjunto de classes abstratas e concretas, ou seja,  uma estrutura parcialmente completa projetada para ser instanciada. Existe também \textit{frameworks}  que disponibilizam \textit{templates} como base para iniciar o desenvolvimento.

Os \textit{frameworks} disponibilizam para os desenvolvedores um vasto conjunto de bibliotecas, permitindo assim a realização de operações de grande porte. Sendo assim, os programadores focam mais em resolver problemas do que reescrever códigos, aumentando consideravelmente a produtividade da equipe. 