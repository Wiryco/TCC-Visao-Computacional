\subsubsection{\textit{Git}}

Basicamente, o \textit{Git} é um sistema distribuído e de código aberto que controla versões de arquivos desenvolvidos, no qual possui diversos comandos que auxilia os desenvolvedores a realizar projetos de grade e pequeno porte com velocidade e eficiência.  Com esses comandos, o \textit{Git} disponibiliza um amplo conjunto de ferramentas para realizar a manipulação dos arquivos no seu repositório local ou \textit{Web}, garantindo a integridade dos dados \cite{GIT2010}.

Segundo \citeonline{CHACON2014}, o maior diferencial do sistema de controle de versões \textit{Git} é o seu modelo de ramificação (\textit{branch}), que possibilita o usuário a criar uma cópia do sistema principal. Com essa cópia, o desenvolvedor pode realizar implementações de melhorias/correções em trechos de códigos sem modificar a aplicação principal. Depois da implementação e dos testes, o desenvolvedor pode substituir a \textit{branch} principal (\textit{master}) pela \textit{branch} com implementações. Os únicos arquivos que serão substituídos serão os arquivos editados. 