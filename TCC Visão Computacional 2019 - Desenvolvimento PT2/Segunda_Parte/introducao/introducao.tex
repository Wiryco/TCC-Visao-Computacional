\chapter{Introdução}
\label{chapter:introducao}

A evolução tecnológica expandiu consideravelmente nos dias atuais, proporcionando um avanço descomunal de vários hardwares poderosíssimos capazes de realizar tarefas jamais vistas pelos gênios antigos da computação. Devido a isso, a imersão de varias técnicas vem sendo estudadas progressivamente nos últimos anos, visto que atualmente existem equipamentos capazes de realizá-las em um curto prazo de tempo e de maneira mais eficaz.

A área de visão computacional se tornou de grande interesse nos tempos atuais devido a sua complexidade ao realizar processamentos de imagens e extrair o maior número de informações desta. O campo de processamento de imagem complementa esta parte, utilizando técnicas matemáticas e probabilísticas para corrigir os parâmetros visuais da imagem que será processada.

Por mais que existam alguns problemas nesta área como, por exemplo, a captura de imagens, processamento, alta precisão, similaridade, dentre outros, os estudos buscam aprimorar a ferramenta para que o campo de visão computacional se aproxime cada vez à eficiência de uma visão humana.


%\chapter{TEMA}
\label{chapter:introducao}

Na atualidade existem vários tipos de segmentos nos quais as tecnologias de visão computacional estão sendo aplicados: câmeras de \textit{smartphones} que detectam a diferença entre cachorro e gato, meios de segurança que escaneiam faces, análise de radiografias no campo da medicina para o auxílio na detecção de doenças e possíveis fraturas, dentre outros.

Dentro do meio desportivo, a tecnologia pode ser aplicada para o reconhecimento de jogadores com a finalidade de realizar análises detalhadas de táticas utilizadas dentro de campo. Sendo assim, os detalhes de cada jogador podem ser extraídos para mais informações deste.

Este trabalho tem por finalidade apresentar, de forma didática, a utilização da tecnologia de visão computacional dentro do esporte de futebol americano, buscando reconhecer um jogador em campo. A utilização da biblioteca \textit{OpenCV}, \textit{Open Source Computer Vision Library} (Biblioteca de visão computacional de código aberto), combinado a linguagem de programação \textit{Python} será aplicada para o desenvolvimento da ferramenta. Portanto, o tema deste trabalho está compreendido no domínio da visão computacional e na área de análise de jogadores de futebol americano.


\input{Segunda_Parte/introducao/objetivo-intro.tex}
\section{JUSTIFICATIVA}

A área de visão computacional tem crescido de forma significativa no mundo presente, devido ao avanço tecnológico. Várias informações são capturadas e o volume de dados encontram-se crescendo progressivamente. Sendo assim, várias aplicações que utilizam a tecnologia de visão computacional estão sendo criadas de forma singular e concisa.

Este crescimento ocorre principalmente devido ao aumento da utilização de dispositivos móveis. Não é difícil se deparar com vários equipamentos que já utilizam essa tecnologia para alguma função, seja ela para melhorar algo ou para realizar a identificação de algum objeto ou biometria de segurança, como por exemplo o \textit{Face ID} (Identidade de Rosto) da \textit{Apple} e o \textit{Google Lens}.

Sendo assim, a visão computacional pode auxiliar na questão relacionada ao reconhecimento de um jogador em uma partida de futebol americano.

As jogadas realizadas em futebol americano são de total contato físico e de alta velocidade. As transições dos jogadores são feitas em vários momentos para que as jogadas certas possam acontecer de acordo com a tática traçada pelo técnico. A leitura dessas substituições rápidas são feitas a olho humano, onde podem ocorrer equívocos e possíveis erros na identificação dos jogadores.
\section{ORGANIZAÇÃO DO TRABALHO}

Este trabalho é composto por cinco capítulos, estruturados da seguinte forma: o capítulo 2 é composto pelos fundamentos conceituais que forma necessários para o entendimento da área de visão computacional. No capítulo 3 foi abordado a metodologia utilizada para a construção deste trabalho.  Subsequente, o capitulo 4 relata todo o desenvolvimento do trabalho, ressaltando todas as etapas de funcionamento, descrição do sistema e requisitos. Por fim, o capítulo 5 apresenta as considerações finais obtidas com o projeto, seguida das análises dos resultados e possíveis estudos futuros.